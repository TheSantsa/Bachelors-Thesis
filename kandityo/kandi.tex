%%%%%%%%%%%%%%%%%%%%%%%%%%%%%%%%%%%%%%%%%%%%%%%%%%%%%%%%%%%%%%%%%%%%
%%%%%%%%%%%%%%%%%%%%%%%%%%%%%%%%%%%%%%%%%%%%%%%%%%%%%%%%%%%%%%%%%%%%
%%                                                                %%
%% Esimerkki opinnäytteen tekemisestä LaTeX:lla                   %%
%% Alkuperäinen versio Luis Costa,  muutokset Perttu Puska        %%
%% Ruotsinkielen tuki lisätty 15092014                            %%
%%                   	                                             %%
%% An example for writting your thesis using LaTeX                %%
%% Original version by Luis Costa,  changes by Perttu Puska       %%
%% Support for Swedish added 15092014                             %%
%%                                                                %%
%% Tähän esimerkkiin kuuluu tiedostot                             %%
%% This example consists of the files                             %%
%%         opinnaytepohja.tex (versio 2.0)                        %%
%%         thesistemplate.tex (versio 2.0) (for text inEnglish)   %%
%%         aaltothesis.cls (versio 2.0)                           %%
%%         kuva1.eps                                              %%
%%         kuva2.eps                                              %%
%%         kuva1.pdf                                              %%
%%         kuva2.pdf                                              %%
%%                                                                %%
%%                                                                %%
%% Kääntäminen joko                                               %%
%% Typeset either with                                            %%
%% latex:                                                         %%
%%             $ latex opinnaytepohja                             %%
%%             $ latex opinnaytepohja                             %%
%%                                                                %%
%%   Tuloksena on tiedosto opinnayte.dvi, joka                    %%
%%   muutetaan ps-muotoon seuraavasti                             %%
%%   Result is the file opinnayte.dvi, which                      %%
%%   is converted to ps format as follows:                        %%
%%                                                                %%
%%             $ dvips opinnaytepohja -o                          %%
%%                                                                %%
%%   ja edelleen pdf-muotoon seuraavasti                          %%
%%   and then to pdf as follows:                                  %%
%%                                                                %%
%%             $ ps2pdf opinnaytepohja.ps                         %%
%%                                                                %%
%% Tai                                                            %%
%% Or                                                             %%
%% pdflatex:                                                      %%
%%             $ pdflatex opinnaytepohja                          %%
%%             $ pdflatex opinnaytepohja                          %%
%%                                                                %%
%%   Tuloksena on tiedosto opinnaytepohja.pdf                     %%
%%   Result is the file opinnaytepohja.pdf                        %%
%%                                                                %%
%% Selittävät kommentit on tässä esimerkissä varustettu           %%
%% %%-merkeillä ja muutokset, joita käyttäjä voi tehdä,           %%
%% on varustettu %-merkeillä                                      %%
%% Explanatory comments in this example begin with                %%
%% the characters %%, and changes that the user can make          %%
%% with the character %                                           %%
%%                                                                %%
%%%%%%%%%%%%%%%%%%%%%%%%%%%%%%%%%%%%%%%%%%%%%%%%%%%%%%%%%%%%%%%%%%%%
%%%%%%%%%%%%%%%%%%%%%%%%%%%%%%%%%%%%%%%%%%%%%%%%%%%%%%%%%%%%%%%%%%%%

%% Käytä toinen näistä:
%% ensimmäinen, jos käytät pdflatexia, joka kääntää tekstin suoraan 
%% pdf-tiedostoksi (kuvat on oltava jpg- tai pdf-tiedostoina)
%% toinen, jos haluat tuottaa ps-tiedostoa (käytä eps-formaattia kuville,
%% alä käytä ps-muotoisia kuvia!)
%%
\documentclass[finnish,12pt,a4paper,pdftex,elec,utf8]{aaltothesis}
%\documentclass[finnish,12pt,a4paper,dvips]{aaltothesis}

%% Kirjoita y.o. \documentclass optioiksi
%% korkeakoulusi näistä: arts, biz, chem, elec, eng, sci
%% editorisi käyttämä merkkikoodaustapa: utf8, latin1
%%

%% Käytä näitä, jos kirjoitat englanniksi. Katso englanninokset tiedostosta
%% thesistemplate.tex.
%%
%% Uncomment one of these, if you write in English:
%% the 1st when using pdflatex, which directly typesets your document in
%% pdf format (use jpg or pdf figures), or
%% the 2nd when producing a ps file (use eps figures, don't use ps figures!).
%\documentclass[english,12pt,a4paper,pdftex,elec,utf8]{aaltothesis}
%\documentclass[english,12pt,a4paper,dvips]{aaltothesis}

%% To the \documentclass above
%% specify your school: arts, biz, chem, elec, eng, sci
%% specify the character encoding scheme used by your editor: utf8, latin1

%%
\usepackage{graphicx}

%% Matematiikan fontteja, symboleja ja muotoiluja lisää, näitä tarvitaan usein 
%%
%% Use this if you write hard core mathematics, these are usually needed
\usepackage{amsfonts,amssymb,amsbsy}

%% Jos et jostain syystä pidä, miten alla oleva hyperref-paketti käyttää
%% fontteja, värejä yms., käytä tämän paketin makroja muuttamaan
%% fonttimäärittelyt. Katso paketin dokumentaatiota. Paketti määrittelee
%% \url-makron, joten ota paketti käyttöön, jos et käytä hyperref-pakettia.
%%
%\usepackage{url}

%% Saat pdf-tiedoston viittaukset ja linkit kuntoon seuraavalla paketilla.
%% Paketti toimii erityisen hyvin pdflatexin kanssa. 
%%
\usepackage{hyperref}
\hypersetup{pdfpagemode=UseNone, pdfstartview=FitH,
  colorlinks=true,urlcolor=red,linkcolor=blue,citecolor=black,
  pdftitle={Default Title, Modify},pdfauthor={Your Name},
  pdfkeywords={Modify keywords}}


%% Kaikki mikä paperille tulostuu, on tämän jälkeen
%%
%% All that is printed on paper starts here
\begin{document}

%% Korjaa vastaamaan korkeakouluasi, jos automaattisesti asetettu nimi on 
%% virheellinen 
%%
%% Change the school field to specify your school if the automatically 
%% set name is wrong
% \university{aalto-yliopisto}
% \school{Sähkötekniikan korkeakoulu}

%% Vain kandityölle: Korjaa seuraavat vastaamaan koulutusohjelmaasi
%%
\degreeprogram{Automaatio- ja systeemitekniikka}
%%

%% VAIN DI/M.Sc.- JA LISENSIAATINTYÖLLE: valitse laitos, 
%% professuuri ja sen professuurikoodi. 
%%
%\department{Radiotieteen ja -tekniikan laitos}
%\professorship{Piiriteoria}
%\code{S-55}
%%

%% Valitse yksi näistä kolmesta
%%
\univdegree{BSc}
%\univdegree{MSc}
%\univdegree{Lic}

%% Oma nimi
%%
\author{Eero Santamala}

%% Opinnäytteen otsikko tulee tähän ja uudelleen englannin- tai 
%% ruostinkielisen abstraktin yhteydessä. Älä tavuta otsikkoa ja
%% vältä liian pitkää otsikkotekstiä. Jos latex ryhmittelee otsikon
%% huonosti, voit joutua pakottamaan rivinvaihdon \\ kontrollimerkillä.
%% Muista että otsikkoja ei tavuteta! 
%% Jos otsikossa on ja-sana, se ei jää rivin viimeiseksi sanaksi 
%% vaan aloittaa uuden rivin.
%% 
\thesistitle{Taajuusmuuttajien käyttö kaivoksissa}

\place{Espoo}
%% Kandidaatintyön päivämäärä on sen esityspäivämäärä! 
%% 

%% For B.Sc. thesis use the date when you present your thesis. 
\date{1.12.2014}

%% Kandidaattiseminaarin vastuuopettaja tai diplomityön valvoja.
%% Huomaa tittelissä "\" -merkki pisteen jälkeen, 
%% ennen välilyöntiä ja seuraavaa merkkijonoa. 
%% Näin tehdään, koska kyseessä ei ole lauseen loppu, jonka jälkeen tulee 
%% hieman pidempi väli vaan halutaan tavallinen väli.
%%
\supervisor{TkT\ Pekka Forsman} %{Prof.\ Pirjo Professori}

%% Kandidaatintyön ohjaaja(t) tai diplomityön ohjaaja(t). Ohjaajia saa
%% olla korkeintaan kaksi.
%% 
%\advisor{Prof.\ Pirjo Professori}
\advisor{TkT\ Pekka Forsman}
%\advisor{DI Tina Tutkija}

%% Aaltologo: syntaksi:
%% \uselogo{aaltoRed|aaltoBlue|aaltoYellow|aaltoGray|aaltoGrayScale}{?|!|''}
%% Logon kieli on sama kuin dokumentin kieli
%%
\uselogo{aaltoRed}{''}

%% Tehdään kansilehti
%%
\makecoverpage


%% Suomenkielinen tiivistelmä
%% Kaikki tiivistelmässä tarvittava tieto (nimesi, työnnimi, jne.) käytetään
%% niin kuin se on yllä määritelty.
%% Tiivistelmän avainsanat
%%
\keywords{Avainsanoiksi valitaan kirjoituksen sisältöä keskeisesti kuvaavia
käsitteitä}
%% Tiivistelmän tekstiosa
\begin{abstractpage}[finnish]
 Tiivistelmä suomeksi.
\end{abstractpage}

%% Pakotetaan uusi sivu varmuuden vuoksi, jotta 
%% mahdollinen suomenkielinen ja englanninkielinen tiivistelmä
%% eivät tule vahingossakaan samalle sivulle
%%
\newpage
%
%% English abstract. Delete if you don't need it. 
%%
%% Opinnäytteen ostikko englanniksi. Poista, jos et tarvitse sitä.
\thesistitle{Thesis template}
%\supervisor{Prof.\ Pirjo Professori}
\advisor{D.Sc.\ (Tech.) Olli Ohjaaja}
%\advisor{M.Sc.\ Polli Pohjaaja}
\degreeprogram{Electronics and electrical engineering}
\department{Department of Radio Science and Technology}
\professorship{Circuit theory}
%% Abstract keywords
\keywords{Resistor, Resistance,\\ Temperature}
%% Abstract text
\begin{abstractpage}[english]
Abstract in English. 
\end{abstractpage}

%% Force new page so that the Swedish abstract starts from a new page
\newpage
%
%% Swedish abstract. Delete if you don't need it. 
%% 
%% Opinnäytteen ostikko ruotsiksi.
\thesistitle{Arbetets titel}
%\supervisor{Prof.\ Pirjo Professori}
\advisor{TkD\ Olli Ohjaaja} %
%\advisor{M.Sc.\ Tina Tutkija}
\degreeprogram{Elektronik och elektroteknik}
\department{Institutionen för radiovetenskap och -teknik}%
\professorship{Kretsteori}  %
%% Abstract keywords
\keywords{Nyckelord p\aa{} svenska,\\ Temperatur}
%% Abstract text
\begin{abstractpage}[swedish]
På svenska
\end{abstractpage}

%% Note that if you are writting your master's thesis in English, place
%% the English abstract first followed by the possible Finnish abstract

%% Esipuhe 
%%
\mysection{Esipuhe}

Esipuhe tähän lel.\\

\vspace{5cm}
Otaniemi, 1.12.2014

\vspace{5mm}
{\hfill Eero H.\ Santamala \hspace{1cm}}

%% Pakotetaan varmuuden vuoksi esipuheen jälkeinen osa
%% alkamaan uudelta sivulta
\newpage


%% Sisällysluettelo
\thesistableofcontents


%% Symbolit ja lyhenteet
\mysection{Symbolit ja lyhenteet}

\subsection*{Symbolit}


\subsection*{Operaattorit}



\subsection*{Lyhenteet}

\begin{tabular}{ll}
AC         & vaihtovirta \\
DC         & tasavirta \\
\end{tabular}


%% Sivulaskurin viilausta opinnäytteen vaatimusten mukaan:
%% Aloitetaan sivunumerointi arabialaisilla numeroilla (ja jätetään
%% leipätekstin ensimmäinen sivu tyhjäksi, 
%% ks. alla \thispagestyle{empty}).
%% Pakotetaan lisäksi ensimmäinen varsinainen tekstisivu alkamaan 
%% uudelta sivulta clearpage-komennolla. 
%% clearpage on melkein samanlainen kuin newpage, mutta 
%% flushaa myös LaTeX:n floatit 
%% 
\cleardoublepage
\storeinipagenumber
\pagenumbering{arabic}
\setcounter{page}{1}


%% Leipäteksti alkaa
%%
\section{Johdanto}

%% Ensimmäinen sivu tyhjäksi
%% 
\thispagestyle{empty}
Johdantotekstiä
\clearpage

\section{Kaivosteollisuuden erityispiirteet}


\subsection{Kaivostyypit}
Avolouhokset, kaivannot
\subsection{Ympäristöolosuhteet}
Kosteaa, pölyä, hiekkaa, kaasuja, lämmintä
\subsection{Sähkönjakelu}
Kaivoksen sähköverkko (EMC häiriöt)
\subsection{Kaivosteollisuuden standardit}
???
\subsection{Koneiston sijoittelu}
Pitkät kaapelit jne.
\clearpage


\section{Taajuusmuuttajien käyttökohteet kaivoksissa}

\subsection{Taajuusmuuttajatyypit}

\subsection{Sovellukset ja toiminnallisuus}
\subsubsection{Kaivinkoneet}
\subsubsection{Liukuhihnat}
\subsubsection{Hissit}
\subsubsection{Tuulettimet ja ilmanvaihto}
\subsubsection{Pumput}



\clearpage

\section{Kaivosympäristön vaatimukset taajuusmuuttajalle}
\subsection{Kotelointi}
\subsection{Jäähdytys ja lämmönsieto}
\subsection{Turvallisuus}
\subsection{Etävalvonta ja -ohjaus}
\subsection{Käyttöikä ja -luotettavuus}
\subsection{Standardit}



\clearpage

\section{Yhteenveto}
%\section{Summary} 

\clearpage

%% Lähdeluettelo
\thesisbibliography
\begin{thebibliography}{99}

\end{thebibliography}

%% Liitteet
\clearpage

\thesisappendix
\section{Liite 1\label{LiiteA}}

\end{document}
